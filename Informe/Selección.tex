\section{Selección de los estándares a seguir}

\paragraph{}A continuación se presentan, a modo de lista no numerada, todos los estándares que hemos decidido aplicar a nuestro proyecto:

\begin{itemize}
	\item \textbf{DCL51-CPP}. No declare ni defina un identificador reservado.
	\item \textbf{DCL52-CPP}. Nunca califique un tipo de referencia con constante o volátil.
	\item \textbf{DCL59-CPP}. No defina un espacio de nombres sin nombre de encabezado.
	\item \textbf{MEM51-CPP}. Desasignar correctamente la memoria asignada a los objetos dinámicamente.
	\item \textbf{MEM52-CPP}. Detectar errores de asignación de memoria.
	\item \textbf{FIO50-CPP}. No ingrese y elabore alternativamente desde un flujo sin una llamada de posicionamiento interviniente.
	\item \textbf{FIO51-CPP}. Cerrar los archivos cuando ya no sean necesarios.
	\item \textbf{ERR51-CPP}. Manejar todas las excecpciones.
	\item \textbf{OOP53-CPP}. Escribir los inicializadores de miembros de los constructores en el orden canónico.
	\item \textbf{MSC52-CPP}. Las funciones que devuelven un valor deben devolver un valor desde todas las rutas de salida.
	
\end{itemize}