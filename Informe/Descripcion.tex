\section{Introducción}

	\paragraph{}La aplicación seleccionada para el desarrollo de la primera práctica de calidad del software se trata de una aplicación destinada a la gestión de una biblioteca.
	
	\paragraph{}Esta aplicación se desarrollo para una de las prácticas de la asignatura Estructuras de datos, impartida en el segundo curso del grado de Ingeniería informática ofertado en la Universidad de Jaén. Los alumnos que realizaron esta práctica fueron Esteban Jódar Pozo (integrante de este grupo de prácticas) y Julian Yopis Ruiz.
	
	\paragraph{}Esta aplicación permite dar de alta a usuarios, los cuales pueden registrarse y hacer pedidos de libros tanto por temática, nombre o ISBN. Además, tiene un esquema de administrador, en el cual se podrá controlar aspectos relativos a pedidos, usuarios, libros que han solicitado los usuarios, etc.
	
	\paragraph{}Por tanto, tiene dos esquemas de entrada, de Administrador y de Usuario.
	
	\paragraph{}El administrador, una vez introducida su clave, tendrá acceso a:
	
	\begin{itemize}
		\item Crear pedidos para la biblioteca y tramitarlos.
		\item Cerrar dichos pedidos una vez finalizados.
		\item Ver los pedidos que tenga pendiente un usuario en concreto y tramitarlos.
	\end{itemize}

	\paragraph{}Y un usuario, si no está registrado lo puede hacer, y si ya lo está:
	
	\begin{itemize}
		\item Puede introducir login y contraseña.
		\item Consultar un libro.
		\item Realizar un pedido.
	\end{itemize}

	\paragraph{}Las estructuras de datos principales que sirven de soporte a la aplicación son listas simples enlazadas tanto de usuarios, como de libros, como de pedidos de usuario y pedidos de biblioteca.
	
	\newpage
	
